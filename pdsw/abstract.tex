\begin{abstract} 

HPC platforms are capable of generating huge amounts of metadata about different entities including jobs, users, and files etc. \textit{Simple metadata}, which describe the attributes of these entities has already been well recorded and used in current systems, like the file size, name and permission mode. However, only a limited amount of \textit{rich metadata}, which record not only the attributes of entities, but also relationships between them, are captured in current HPC systems. The main challenge is that the rich metadata can include huge amounts of data from many sources, including users and applications, and generally must be combined to present a correct view for later query and processing. So, collecting, integrating, processing, and querying such rich metadata place a huge pressure on HPC systems. In this paper, we propose a rich metadata management approach that unifies metadata into one generic and flexible property graph. We argue that this approach both support simple metadata operations, including directory traversal, permission validation, and, more importantly, support rich metadata processing, like provenance storage and query. The benefits of this approach come from the unified way of managing all metadata and also from the rapid evolving graph storage and processing techniques.

%For example, the rich metadata \textit{provenance}, which records the entire life cycle of data objects, is not well supported, even given the fact that provenance provides many appealing data management functionalities, such as determining the quality of the data set, finding the source of data corruption, and tracing all data dependency etc. 

\end{abstract}
