\documentclass[10pt, conference]{IEEEtran}
\usepackage[numbers]{natbib}
%\usepackage{flushend}
%\usepackage{cite}
%\usepackage{enumitem}
\usepackage{color}
\usepackage{xcolor}
\usepackage{tabularx, color, colortbl}
\usepackage{graphicx, subfigure}
\usepackage{amsmath, amssymb}
\usepackage{algorithm}
%\usepackage{algorithmicx}
\usepackage{algpseudocode}
\usepackage{listings}
\usepackage[noblocks]{authblk}
\usepackage{epstopdf}

\definecolor{Gray}{gray}{0.9}
\hyphenation{op-tical net-works semi-conduc-tor}
\algrenewcommand{\algorithmiccomment}[1]{\hskip3em$\rightarrow$ #1}

% Settings for code listing
\lstdefinestyle{customc}{
  breaklines=true,
  frame=none,
  xleftmargin=\parindent,
  language=Java,
  showstringspaces=false,
  basicstyle=\ttfamily\footnotesize,
  keywordstyle=\bfseries\color{green!40!black},
  commentstyle=\itshape\color{purple!40!black},
  identifierstyle=\color{blue},
  stringstyle=\color{orange},
}
\lstset{ %
  style=customc,
  backgroundcolor=\color{white},   % choose the background color; you must add \usepackage{color} or \usepackage{xcolor}
  basicstyle=\ttfamily\footnotesize, % the size of the fonts that are used for the code
  breakatwhitespace=false,         % sets if automatic breaks should only happen at whitespace
  breaklines=true,                 % sets automatic line breaking
  captionpos=b,                    % sets the caption-position to bottom
  %commentstyle=\color{mygreen},    % comment style
  deletekeywords={...},            % if you want to delete keywords from the given language
  escapeinside={\%*}{*)},          % if you want to add LaTeX within your code
  extendedchars=true,              % lets you use non-ASCII characters; for 8-bits encodings only, does not work with UTF-8
  frame=none,                    % adds a frame around the code
  keepspaces=true,                 % keeps spaces in text, useful for keeping indentation of code (possibly needs columns=flexible)
  keywordstyle=\color{black},       % keyword style
  language=Java,                      % the language of the code
  morekeywords={*,...},            % if you want to add more keywords to the set
  numbers=left,                    % where to put the line-numbers; possible values are (none, left, right)
  numbersep=5pt,                   % how far the line-numbers are from the code
  %numberstyle=\tiny\color{mygray}, % the style that is used for the line-numbers
  rulecolor=\color{black},         % if not set, the frame-color may be changed on line-breaks within not-black text (e.g. comments (green here))
  showspaces=false,                % show spaces everywhere adding particular underscores; it overrides 'showstringspaces'
  showstringspaces=false,          % underline spaces within strings only
  showtabs=false,                  % show tabs within strings adding particular underscores
  stepnumber=2,                    % the step between two line-numbers. If it's 1, each line will be numbered
  stringstyle=\color{black},     % string literal style
  tabsize=2,                       % sets default tabsize to 2 spaces
  %title=\lstname                   % show the filename of files included with \lstinputlisting; also try caption instead of title
  aboveskip=1em,
  belowskip=1em,
  xleftmargin=0.5em,
  xrightmargin=0.5em,
}
\begin{document}
\title{gRMM: A Unified Graph-Based Rich Metadata Model for HPC Platform}

%\author[1]{Dong Dai}
%\author[1]{Yong Chen}
%\author[2]{Phil Carns}
%\author[2]{Robert Ross}
%\author[2]{Dries Kimpe}
%\affil[1]{Computer Science Department, Texas Tech University, USA, \{dong.dai, yong.chen\}@ttu.edu}
%\affil[2]{Mathematics and Computer Science Division, Argonne National Laboratory, USA, \{dkimpe, pcarns, rross\}@mcs.anl.gov}

\maketitle
\begin{abstract} 

HPC platforms are capable of generating huge amounts of metadata about different entities including jobs, users, and files etc. \textit{Simple metadata}, which describe the attributes of these entities has already been well recorded and used in current systems, like the file size, name and permission mode. However, only a limited amount of \textit{rich metadata}, which record not only the attributes of entities, but also relationships between them, are captured in current HPC systems. The main challenge is that the rich metadata can include huge amounts of data from many sources, including users and applications, and generally must be combined to present a correct view for later query and processing. So, collecting, integrating, and processing such rich metadata place a huge pressure on HPC systems. In this paper, we propose a rich metadata management approach that unifies metadata into one generic and flexible property graph. We argue that this approach can be leveraged to efficiently support simple metadata operations, including directory traversal, permission validation, and, more importantly, support rich metadata processing, like provenance storage and query. The benefits of this approach come from the unified way of managing all metadata and also from the rapid evolving graph storage and processing techniques.

%For example, the rich metadata \textit{provenance}, which records the entire life cycle of data objects, is not well supported, even given the fact that provenance provides many appealing data management functionalities, such as determining the quality of the data set, finding the source of data corruption, and tracing all data dependency etc. 

\end{abstract}

\section{Introduction}

Metadata, especially rich metadata, contains the detailed information of different entities and their relationships in HPC. These entities could be users, jobs, processes, data files, and even user-defined entities. Storing and utilizing the metadata has already provided the basic data management functionalities in most existing storage systems, including finding files, controlling file accessing, and tracing file creation and accessing time etc. We category this part of metadata as the \textit{simple metadata} since they only contain the predefined attributes about individual entity. As a contrary, \textit{rich metadata} cares more than individual predefined attributes; it may store the user-defined arbitrary attributes of entities and even their relationships. A typical example of rich metadata would be provenance (e.g. lineage). 

Provenance is well understood in the context of art or digital libraries, where it respectively refers to the documented history of an art object, or the documentation of processes in a digital object's life cycle. In the computational systems, it indicates a recording of complete history of each data piece, including the processes that generated it, the users that started the processes, and even the environment variables, parameters, and configuration files while executing. A complete provenance information supports huge amount of data management abilities. For example, the  accessing history of users reading/writing data files can help us develop a audit tool to monitor and administrate users in shared supercomputer facilities; the detailed read/write history from processes to data pieces provides a possibility to trace back the suspicious executions that generated or were based on wrong datasets; reproducibility also may be possible because we have the complete history of an execution and have a better chance to re-generate the same environment to run it again. 

With such greatness of rich metadata like provenance, current HPC platforms still lack of providing basic facility to collect, store and process the rich metadata. The challenge comes from at least three places.

\begin{itemize}

\item \textit{Storage Pressure}. Considering a leadership supercomputer, there might be millions of processes running on millions of cores accessing billions of files per second. In this case, recording the rich metadata, like detailed accessing history of each process will place a huge pressure on the storage systems. In addition, as storing rich metadata should not affect the application execution speed significantly, the resources (both network and disks bandwidth) dedicated to storing metadata are limited to use in most cases.

\item \textit{Efficient Processing}. Even we already have the perfect collected and stored rich metadata, it is still a big challenge to process them. First, as the rich metadata is large and can not be hold in one server, the distributed processing framework is necessary in most cases. Second, many use cases require complex analysis instead of simple searching or reading, so flexible processing should be provided for them.

\item \textit{Metadata diversity}. As we have described, the rich metadata could be as diverse as the usages need. They can only contain predefined attributes and relationships of entities, or be extended to any user-defined attributes and relationships. Traditionally, we used different tools (system components or users applications) to store and process part of metadata based on the specific usages. However, this introduces lots of unnecessary redundant metadata storage in different tools. For example, both the data audit tools and data verification tools need to store the file access history of users, so this metadata usually was stored twice in two applications. If we only store each metadata once in one application or component, then there will be massive cross-reference operations between them during later processing or analysis, which is inefficient.

\end{itemize}

In this paper, we proposed a idea of unifying all metadata (\textit{simple} and \textit{rich}) into a general graph-based model for HPC platform. By exploring the collection, storage, and processing of metadata, we form a practical solution named \textit{gRMM} introduced in this paper to provide such unified metadata service. 

\textit{gRMM} integrates itself closely into HPC storage systems to store rich metadata in an efficient and consistent way. In our case, the underlying storage system is Triton. The data model for storing is based on graph abstraction: all metadata for entities, attributes and relationships was abstracted as elements in a unified graph model and stored by calling the graph-based storage APIs. Moreover, by closely integrating into storage systems, \textit{gRMM} is able to trace all reads/writes on data pieces in storage system and provides detailed metadata for these data pieces automatically. Users can easily access these metadata later, combine them with information from other components to build rich metadata, and store them. 

More than this, \textit{gRMM} also contains a runtime library to help users query and process existing metadata. All these queries and processing are based on mature graph algorithms, which is expressive and efficient. In general, \textit{gRMM} aims at providing a genetic layer for easing the burden of collecting, storing and managing metadata in a modern HPC system. Working with a fault-tolerant object storage system like Triton, we could form a fully functional parallel file systems using \textit{gRMM} easily.

This paper was organized as follow: in Section II, we will first introduced the rich metadata graph model. In Section III, we introduce different metadata use cases with incremental complexity, showing how to map different use cases into proposed graph model, and solve them using facilities provided by \textit{gRMM}. In Section IV, we introduce system components and discuss the design considerations and challenges in different components. In section V, we conclude the study and list future work.
\section{Graph-based Metadata Model}
In fact, we already consider metadata as a graph. The traditional directory-based file management constructs a tree structure to manage files with additional metadata stored in \textit{inodes} at leaves in the tree. This tree is a graph. The provenance standard (\textit{Open Provenance Model}) considers the provenance of objects is represented by an annotated causality graph, which is a directed acyclic graph enriched with annotations capturing further information. 

We generalize these graphs in HPC scenarios and propose the metadata graph model. The metadata graph is derived from the \textit{property graph model}, which includes vertice that represent entities in the system, edges that show their relationships, and properties. The properties are the main difference from a traditional weighted graph; they annotate both vertice and edges, and can store arbitrary information users want. Based on the entities in HPC environment, we introduce the strategy to map the possibly arbitrary rich metadata into this property graph model.

\subsection{Entity To Vertex}

In an HPC platform, there are different entities (e.g. users, processes, and data files) that play different roles. Moreover, users also can define other logical entities, like \textit{user groups} or \textit{work-flow} as they need. 
So, we define three basic entities, and allow users to extend them as user-defined entities.

\begin{itemize}
\item \textit{Data Object}: It represents the smallest data unit in storage systems. Each file in PFS(Parallel File System) indicates one data object. Moreover, the directory is also a data object, which contains multiple other data objects. The applications or users programs are also data objects.

\item \textit{Executions }: They represents the execution of applications. There are basically three kinds of executions: the \textit{Job} submitted by
the user, parallel \textit{Processes} scheduled from one job, and the possible \textit{Threads} running inside each processes. For simplicity, we name all these entities as \textit{Execution} entity in later discussion.

\item \textit{User}: It represents the real users of the cluster.
\end{itemize}

In addition to these basic entities, users can define their own entities. The only limitation is the new entity must connect with existing entities. The reason is to keep every element in the graph accessible by traveling through the graph.

\subsection{Relation To Edge}
We define several basic relationships based on these basic entities in Table~\ref{rel}. Each cell shows the relationships from the row identifier to the column identifier. It denotes a directed edge in the metadata graph. For example, \textit{run} indicates that the User runs an Execution, which could be Job or just a Process; \textit{exe} means one Execution was started from an application, which are Data Objects.

\begin{table}[h]
\caption{Default Relationships Definition.}
  \label{rel}
\centering
\begin{tabular}{|c||c|c|c|}
\hline
 & \textbf{User} & \textbf{Execution} & \textbf{Data Objects} \\ \hline
\textbf{User} &  & \textit{run} & \\ \cline{1-4}
\textbf{Execution} & \textit{wasRunBy} & \begin{tabular}[c]{@{}l@{}}\textit{belongs,}\\ \textit{contains}\end{tabular} & \begin{tabular}[c]{@{}c@{}}\textit{exe,}\\ \textit{read,}\\ \textit{write}\end{tabular}\\ \cline{1-4}
\textbf{Data Objects} &  & \begin{tabular}[c]{@{}c@{}} \textit{exedBy,} \\ \textit{wasReadBy,}\\ \textit{wasWrittenBy} \end{tabular} & \begin{tabular}[c]{@{}l@{}}\textit{belongs,}\\ \textit{contains}\end{tabular} \\ \hline
\end{tabular}
\end{table}

There are \textit{belongs/contains} cells. In the Execution entity case, it means the Job contains Processes and Process belongs to a Job. In the Data Objects case, it can show that one directory may contain multiple files or directories. Users can create their own relationships, which could be metadata that current graph does not record. For example, two users can have a new relationships called \textit{login-together} if they login the system roughly at the same time.

\subsection{Property}
To collect rich metadata, we always need to collect annotations on entities or their relationships. In proposed graph model, we can store them using properties. which is in a key-value formation and could be default or user-defined. For example, all entities and relationships both have a default property named `Type' to distinguish themselves, like User vertice have `Type' as 'user'. And, there usually are attributes for different entities, like the user name, user privilege, execution parameters, data name, and data permission mode etc. We could abstract them as properties of the graph. Timestamps is another important category of property. For example, the \textit{run} relationship has a $start_{ts}$ and a $end_{ts}$ attribute and the \textit{read} relationship has one $ts$ attribute. 

Users can create their own properties except the keys need to be unique in each users' namespace. We isolate properties by users, so that users are free to create any metadata without introducing global contention.


\section{Use Cases On Using Metadata Graph}

Unifying rich metadata into one graph turns many appealing data management functionalities into graph traversal operations or graph queries. In this section, we will show how to map the use cases from real-world scenarios to graph operations.

\subsection{User Audit}

Data auditing is a critical capability in large computing facilities where users from different institutions or countries share the same cluster. A detailed view of users behaviors can be useful for daily maintenance and security. In metadata graph, we already collect the \textit{run} relationships between Users and Executions, and the \textit{read/write} relationships between Executions and Data Objects. Moreover, all those relationships contain properties like timestamps. In such graph, the need to find all the files that were read by a specific user during given time frame [$t_s$, $t_e$] will become several graph operations like this: 1) query the metadata graph from the given user; 2) travel through \textit{run} edges to Execution nodes; 3) filter executions based on the given time frame; 4) and travel through the \textit{read} edges to the final files. Similarly, if we want to get all the users who used to read to a sensitive file, we can do the similar graph query from the give file node.

%This query can be expressed as a Gremlin script easily like following code shows:
%\begin{lstlisting}
%files = graph.V(userA).out('run')
%		     .filter{it.start_ts > t_s 
%		     		     && it.end_ts < t_e}
%		     .out('read')
%\end{lstlisting}

%Similar, if we want to get all the users that wrote to a file which was broken since $t_s$, we can write Gremlin script like this:
%\begin{lstlisting}
%users = graph.V(fileA).out('wasWrittenBy')
%		     .filter{it.start_ts > t_s}
%		     .out('wasRunBy')
%\end{lstlisting}


\subsection{Hierarchical Data Traversal}
Hierarchical data organization is used to present a logical layout of data sets to users. The simplest example of hierarchical data traversal is traditional directory namespace traveling, which has been the de facto method to travel through file systems for as long as Unix has existed. In metadata graph model, we already abstract both the directories and files as Data Object entities. The \textit{belongs} and \textit{contains} relationships between different Data Objects represent the relationships between files and directories. So, given an absolute path, locating the file becomes going through a bunch of \textit{contains} edges from a Data Object node. Each time, we filter the edges according to the given names. Moreover, the access control metadata attached in users, files, and directories also could be verified while traversing. 

%\begin{lstlisting}
%// locate: /rootFS/dir/file.data
%graph.V(rootFS).out('contains')
%     .filter{it.name = dir}
%     .out('contains')
%     .filter{it.name = file.data}
%\end{lstlisting}

An appealing advantage of using graph to store the directory structure of file system is the scalability. Traditional POSIX directory structure limits the number of files inside one directory. So, HPC system that may have millions of files under one directory, needs to deploy specific system like Giga+ to distribute the metadata into multiple servers for better performance. However, for proposed graph model, this is a genetic graph slicing problem, which we have several standard ways to tackle.

In addition to traditional POSIX-style files and directories, several other hierarchical data traversal examples are possible in graph model too, like the semantic data management. Scientists usually need to manage their data in a semantic way, like arranging all of the inputs and outputs of a single simulation execution together. Traditionally, this needs careful file naming and directories placement. But in metadata graph, we can simply create an entity named Simulation and connect it with the Data Object with \textit{input/output} relationships. This will intelligently help users organize data in multiple dimensions.

\subsection{Provenance Support}
Provenance has a wide range of use cases including data reproducibility, work-flow management etc. As a superset of provenance, the metadata graph model is able to support most of these usages. In this subsection, we borrow the problem from the first Provenance Challenge as an example. 

In this challenge, a simple example work-flow was provided as the basis, a workable provenance system should be able to represent the work-flow and all the relevant provenance for the example work-flow, and, most importantly, be able to answer nine predefined queries. Based on proposed graph model, we can easily abstract the work-flow as serial of Executions run by one User. Each execution reads several Data Objects and generates outputs for applications in next phase. Based on this work-flow, the provenance system needs to answer nine queries. Here, we use the 8th query as an example:

\begin{lstlisting}
//Query 8
wf{*}: upstream(x) union x
       where x.module=AlignWarp
       and
       y in input(x)
       and
       y.annotation('center')='UChicago'
\end{lstlisting}

This query mean we need to return all applications whose module is `AlignWarp' and all their inputs are annotated with key-value pair: [`center':`UChicago']. This can be expressed easily in the metadata graph: 1) query all the nodes that represent Execution entity and have an \textit{exe} out-edge that points to a Data Object named `AlignWarp'; 2) start from all those nodes and filter out those executions whose property `center' does not equal to `UChicago'; 3) return all these executions.
%\begin{lstlisting}
%//Query 8
%exes = graph.V(Executions).out('exe')
%			.filter(it.name = AlignWarp)
%exes.out('read').filter(it.center = 'UChicago')
%\end{lstlisting}

A notable advantage of our metadata graph comparing with pure provenance system is that we can cross-reference different category of metadata in an unified way. If the provenance query needs the help of other metadata, like the file size, permission mode, or user group information etc., processing them in a unified graph will be more efficient and straightforwards.
\section{Graph Facilities \& Design Choices}

The limitation of traditional databases especially the relational databases has lead the development of new categories of system called \textit{graph databases} to cover the requirements of complex graph-based relationships. In the last couple of years, there have been an increasing work about graph databases. Some popular graph databases include Sones, AllegroGraph, DEX, G-Store, HyperGraphDB, InfinitGraphDB, Neo4j, Titan, and OrientDB etc.

\subsection{Graph Databases}
These graph databases can be categorized based on different metrics. Based on data storage, there are in-memory databases (e.g. Sones) and disk-based databases (e.g. AllegroGraph, DEX, Neo4j, Titan). Based on the graph data structure, there are \textit{simple graphs} databases, \textit{hypergraphs} databases, and \textit{property graphs} databases. Here, the simple graph indicates graph defined as a set of nodes connected by weighted edges. AllegroGraph and G-Store only support this simple graph data structure. Hypergraphs extends the simple graphs by allowing an edge to relate an arbitrary number of nodes. Databases like HyperGraphDB and Sones support these hypergraphs. Property graph indicates graph where nodes and edges contain properties. This property graph is the very basic of our proposed metadata graph model. Many databases like Neo4j, Titan, InfinitGraph, and DEX etc. support such graphs. 

In addition to this, some graph databases support query languages like SQL in relational databases. For example, Neo4j supports Cypher graph traversal language, Titan supports Gremlin language etc. Although most graph databases provide basic APIs to travel graphs, it is still more easy and efficient to use a query language. 

Whether support distributed deployment also classify all these databases. Only part of those databases support distributed deployment due to the fact that distributing graph into different servers with ACID and enough performance is challenge. For example, Neo4j only supports high availability deployment, which means multiple copies of each data across the cluster rather than a distributing. There are still lots of research work going on this problem.

\subsection{Storing and Indexing Requirement}

Actually, the graph database choices in our case are limited by the attributes of the metadata graphs. First, the metadata graphs are too large to fit into memory, so, the disk-based databases are better. Moreover, as the metadata graph is based on property graph model, which may store a large amount of properties, the graph could even too large to fit in one server. The distributed supports will be another necessary component. About the query language, it is not the first concern yet, but still, graph databases with a query language and some query optimizer (e.g. indexing) will be much better.  

\subsection{Traversal Requirement}

From the example use cases in previous section, we can see that the key to use the metadata graph to support different use cases is effective graph traversal. In fact, the graph size, graph data structure, and storage layouts, all decided the performance of traveling a graph. The simplest case is to deal with moderate-sized simple graph. In this cases, traveling from one or several , or even all vertices and explore their k-hop neighborhood can be possible to perform in memory as the simple graph is smaller and possible be cached into memory. 

However, traveling through a property graph like our metadata graph will be much complex. The main reason is that, during traveling, we usually need to apply filter or computations on some properties like previous provenance example shows. As these properties are too big to be cached in memory, each time, we need to load them from persistent devices. This will introduce lots of random seeking  leading to a poor performance. This is actually still a big challenge for property graph databases waiting for solving.

\subsection{Graph Processing}

In addition to the graph databases, there are also graph processing frameworks which can be used to perform computation or queries on graphs in a distributed way. Typical examples of these frameworks include Gigraph, which was designed and implemented based on Pregel computing model; GraphX, which was based on Spark computing framework; GraphLab, X-Stream, and GiGraph+ etc. 

However, most these distributed graph processing frameworks work on the unstructured graphs which are usually simply stored as a plain file in CSR or CSC formats in a general storage back-end , like local disk, HDFS, S3, or RDD etc. So, there is a big gap from deploying graph algorithms on these plain graph formats to running these algorithms on a graph database. However, the ability to run complex graph algorithms is necessary for metadata management, and this ability can not be easily satisfied using the querying or searching facilities provided by graph databases. For example, in HPC system, we can run \textit{community discovery} algorithms on metadata graph to find the `closely' data files, which can be used to optimize their physical placements for better future I/O performance. These algorithms get the whole graph involved in the computation, contain multiple iterations, and last a long time. A distributed fault-tolerant graph processing model is a much better choice than writing applications to manage all these complexity.
\section{Conclusion \& Future Work}

In this paper, we proposed an idea of unifying rich metadata in a HPC platform into a property graph model, which supports a wide range of metadata management requirements in a simple and efficient way. By prototyping such a metadata graph from Darshan I/O traces of a real world leading supercomputer, we explorer the attributes of such graphs and compare it with some popular big graphs. After that, we introduce the existed graph facilities and discuss their feasibilities and limitations in our scenario. In general, we argue the benefits of unifying HPC metadata into a graph and also present the feasibility of implementing such a graph in current HPC platform. In future, the work will be implementing such a platform with optimized or tweaked graph facilities and provide a practical metadata solution for Exscale data management challenge. 
\input{ack}

\bibliographystyle{IEEEtran}
\bibliography{bib}

\end{document}
